\documentclass{article}
\usepackage[utf8]{inputenc}
\usepackage[english,russian]{babel}
\usepackage{graphics}
\usepackage{amsmath}
\usepackage{indentfirst}
\usepackage{misccorr}
\usepackage{graphicx}

\begin{document}
	Постройте график функции\\
	\begin{equation*}
		y = 
		\left\{
			\begin{array}{ccc}
				x^2-6x+13, x \geq 2 \\
				2,5x, x < 2\\
			\end{array}
		\right.
	\end{equation*}
	\\
	и определите, при каких значениях m прямая {$y=m$}   будет пересекать построенный график в трёх точках.\\
	\begin{figure}[h]
		\begin{center}
			\includegraphics[scale=0.2]{Graphic}
		\end{center}
	\end{figure}
	\\
	Сначала найдём минимальное значение функции $x^2-6x+13, x \geq 2$:\\
	$x_{0} = -(-6)/2 = 3$, $y_{0} = 3^2-6*3+13=4$\\
	\\
	Далее найдём максимальное значение функции $2,5x, x < 2$:\\
	Так как функция монотонно возрастает, то $y = max$ при $x = max$, значит $x=2$, $y=5$\\
	\\
	Построим 2 прямые: $y = 4$ и $y = 5$:\\
	\begin{figure}[h]
		\begin{center}
			\includegraphics[scale=0.2]{Graphic 1}
		\end{center}
	\end{figure}
	\\
	\\
	\\
	\\
	\\
	\\
	\\
	\\
	\\
	\\
	\\
	Как мы видим если $y>4$ и $y<5$, то прямая $y = m$  имеет с графиком 3 общие точки, а если $y<4$ или $y>5$, то прямая $y = m$  имеет с графиком 1 общую точку\\
	\\
	Ответ: $m = 4$ или $m=5$
\end{document}
